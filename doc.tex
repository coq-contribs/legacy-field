\documentclass{article}

\usepackage[latin1]{inputenc}
\usepackage[T1]{fontenc}
\usepackage{times}
\usepackage{url}
\usepackage{verbatim}
\usepackage{amsmath}
\usepackage{amssymb}
\usepackage{alltt}
\usepackage{fullpage}
\usepackage[linktocpage,colorlinks]{hyperref}

%%%% Excerpt of the RefMan chapter concerning Legacy Field %%%%

\input{../LegacyRing/macros}

\newcommand{\tacindex}[1]{}
\newcommand{\comindex}[1]{}
\newcommand{\errindex}[1]{}
\newcommand{\errindexbis}[2]{}
\newcommand{\ttindex}[1]{}

\title{Legacy implementation of the Field tactic}
\date{}

\begin{document}

\maketitle

This tactic written by David~Delahaye and Micaela~Mayero solves equalities
using commutative field theory. Denominators have to be non equal to zero and,
as this is not decidable in general, this tactic may generate side conditions
requiring some expressions to be non equal to zero. This tactic must be loaded
by {\tt Require Import LegacyField}. Field theories are declared (as for
{\tt legacy ring}) with
the {\tt Add Legacy Field} command.

\subsection{\tt Add Legacy Field
\comindex{Add Legacy Field}}

This vernacular command adds a commutative field theory to the database for the
tactic {\tt field}. You must provide this theory as follows:
\begin{flushleft}
{\tt Add Legacy Field {\it A} {\it Aplus} {\it Amult} {\it Aone} {\it Azero} {\it
Aopp} {\it Aeq} {\it Ainv} {\it Rth} {\it Tinvl}}
\end{flushleft}
where {\tt {\it A}} is a term of type {\tt Type}, {\tt {\it Aplus}} is
a term of type {\tt A->A->A}, {\tt {\it Amult}} is a term of type {\tt
  A->A->A}, {\tt {\it Aone}} is a term of type {\tt A}, {\tt {\it
    Azero}} is a term of type {\tt A}, {\tt {\it Aopp}} is a term of
type {\tt A->A}, {\tt {\it Aeq}} is a term of type {\tt A->bool}, {\tt
  {\it Ainv}} is a term of type {\tt A->A}, {\tt {\it Rth}} is a term
of type {\tt (Ring\_Theory {\it A Aplus Amult Aone Azero Ainv Aeq})},
and {\tt {\it Tinvl}} is a term of type {\tt forall n:{\it A},
  {\~{}}(n={\it Azero})->({\it Amult} ({\it Ainv} n) n)={\it Aone}}.
To build a ring theory, refer to Chapter~\ref{ring} for more details.

This command adds also an entry in the ring theory table if this theory is not
already declared. So, it is useless to keep, for a given type, the {\tt Add
Ring} command if you declare a theory with {\tt Add Field}, except if you plan
to use specific features of {\tt ring} (see Chapter~\ref{ring}). However, the
module {\tt ring} is not loaded by {\tt Add Field} and you have to make a {\tt
Require Import Ring} if you want to call the {\tt ring} tactic.

\begin{Variants}

\item {\tt Add Legacy Field {\it A} {\it Aplus} {\it Amult} {\it Aone} {\it Azero}
{\it Aopp} {\it Aeq} {\it Ainv} {\it Rth} {\it Tinvl}}\\
{\tt \phantom{Add Field }with minus:={\it Aminus}}

Adds also the term {\it Aminus} which must be a constant expressed by
means of {\it Aopp}.

\item {\tt Add Legacy Field {\it A} {\it Aplus} {\it Amult} {\it Aone} {\it Azero}
{\it Aopp} {\it Aeq} {\it Ainv} {\it Rth} {\it Tinvl}}\\
{\tt \phantom{Add Legacy Field }with div:={\it Adiv}}

Adds also the term {\it Adiv} which must be a constant expressed by
means of {\it Ainv}.

\end{Variants}

\SeeAlso \cite{DelMay01} for more details regarding the implementation of {\tt
legacy field}.

\end{document}
